\head{Solution Enhancement}

There are a number of small-scale improvements we can make to the initial solution that will bring benefits in reliability, security, and observability. These improvements are small and iterative but have a large impact on the production performance. 

\architecture
{aws_sa_v1_simple}
{Architecture diagram of an improved solution.}
\changes
{Changes and enhancements to the initial solution (Figure \ref{fig:aws_sa_v1_simple})}
{
	\item Addition of ACM certificate to load balancer to allow HTTPS to be enabled
	\item Moving to Application Load Balancer(ALB) for HTTP->HTTPS redirection and layer 7 routing and WAF filtering
	\item Enable detailed instance and load balancer metrics, shipping to Cloudwatch with dashboard. Log shipping and OS-level metrics through collectd and cloudwatch agent installation on AMI
\item Instances placed in Auto-scaling group (ASG) for automatic management and healing of instances using load balancer HTTP:80 healthcheck 
}

\FloatBarrier

\subhead{Operational}

\textit{The ability to run and monitor systems to deliver business value and to continually improve supporting processes and procedures}

\subsubhead{Improve monitoring and logging}

Monitoring and logging on the existing instance can be increased by enabling the cloudwatch agent on the application instances


\subhead{Security}

\textit{The ability to protect information, systems, and assets while delivering business value through risk assessments and mitigation strategies}

\subsubhead{Harden EC2 instances}

\subsubhead{Enable SSM session manager}

\subsubhead{Strengthen NACL to prevent cross-subnet traffic}

\subsubhead{Only allow NACL to pass traffic to private subnets from public subnets}

The private NACL allows all external traffic to pass. We can increase security by allowing only access from public subnets, or additionally from private endpoints, NAT gateways etc.

\subhead{Reliability}

\textit{The ability of a system to recover from infrastructure or service disruptions, dynamically acquire computing resources to meet demand, and mitigate disruptions such as misconfigurations or transient network issues}

Currently there is a single instance with no auto-recovery

\subsubhead{Replicate instances into multiple AZs}

\subsubhead{Create an auto-scaling group for application instances}

\subhead{Performance}

\textit{The ability to use computing resources efficiently
	to meet system requirements, and to maintain that efficiency as demand changes and technologies evolve}

\subsubhead{Change instance type}

The current solution uses bursting t2.micro instances with CPU credit limits. This is very cost efficient, but performance is not suitable for reasonable traffic levels and exhausting CPU credits will cause degradation of availability.

\subhead{Cost} 

\textit{The ability to run systems to deliver business value at the lowest price point}





