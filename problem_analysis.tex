\FloatBarrier

\section{Solution analysis}

The current implementation cannot be accessed from external networks. The service consists of,

\items{
	\item Networking is a VPC with CIDR $10.0.0.0/16$ with 2 public subnets $10.0.0.0/24$ and  $10.0.1.0/24$ in eu-west-1b and eu-west-1b respectively, as well as 2 private subnets $10.0.2.0/24$, $10.0.3.0/24$ in eu-west-1b and eu-west-1b. 
	
	\item VPC has an internet gateway and a classic load balancer (ELB)
	 
	\item Routing is setup up to allow all traffic between subnets and out to external addresses.
	
	\item There are 2 security groups, one for the single EC2 application instance, and one for the ELB.
	 
	\item DNS is enabled and the instane has a pubic IP address
}

\imagefigsingle
{aws_sa_v0_simple}
{My explanation}

\FloatBarrier

\issue{ELB healthcheck on wrong instance port}
{Change ELB port for healthcheck from 443 to 80}
{
	Httpd starts on-instance on port 80. The ELB correctly forwards ports
	from from 80 to instance port 80.  However, the healthcheck looks to verify instance health on port 443 through a tcp connection. This change simply targets the healthcheck correctly at port 80 on the instance.
}
{
	\imagefigdual
	{lb-healthcheck_443}
	{In the AWS console, navigate to Services  $\rightarrow$ EC2  $\rightarrow$ Load Balancers}
	{lb-healthcheck_80}
	{In the 'Health check' tab, and change the port to $80$}
}

\FloatBarrier


\issue{Load balancer cannot route to subnet}
{Add PublicSubnetA to load balancer}
{
	The ELB can only route to instances in PublicSubnetB. This change
	allows it to route to PublicSubnetA also, where the current instance is located.
}
{
	\imagefigdual
	{lb-missing_subnet}
	{In the Load Balancers configuration page, select the ELB}
	{lb-subnet_added}
	{Add PublicSubnetA to the load balancers available subnets}
}

\FloatBarrier

\issue{The site cannot be accessed from external addresses}
{Add everywhere access to load balancer ingress}
{
Allow all addresses to access the load balancer on port 80
}
{
	\imagefigdual
	{sg-elb_no_rules_full}
 	{ Navigate to Services  $\rightarrow$ EC2  $\rightarrow$ Security Groups. Select the ELBSecurityGroup group and the 'Inbound' tab}
	{sg-elb_external_access}
	{Edit the group and add an inbound rule to allow http access from everywhere, $0.0.0.0/0$}
}


\FloatBarrier


\issue{Instances cannot be accessed from the load balancer}
{Allow access to the instance from the ELB only}
{This change allows traffic from the ELB to the instances. This is set to all TCP traffic to port 80 to allow the ELBs TCP-based healthcheck to pass}
{
{
	\imagefigdual
	{sg-app_no_rules_full}
	{Select the AppServerSecurityGroup group and the 'Inbound' tab}
	{sg-allow_instance_access_from_elb}
	{Edit the group and add an inbound rule to allow http access from the ELBSecurityGroup}
}
}

\FloatBarrier
