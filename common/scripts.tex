\usepackage{enumitem}
\usepackage{xcolor}
\usepackage{mdframed}

\usepackage{graphicx}
\usepackage{caption}
\usepackage{placeins}
\usepackage{tabularx}

% Generic colours
\definecolor{comments}{gray}{0.95}
\definecolor{actions}{rgb}{1,0.5,0}
\definecolor{done}{rgb}{0.8,0.8,0.8}
\definecolor{revision}{rgb}{0.87,0.97,0.98}
\definecolor{notes}{rgb}{0.5,0.8,0.8}
\definecolor{alerts}{rgb}{1.0, 0.5, 0.5}

% Generic commands
\newcommand{\todobox}[1]{\todo[inline,color=white]{#1}}
\newcommand{\tododone}[1]{\todo[inline,color=done]{DONE: #1}}
\newcommand{\todonote}[1]{\todo[inline,color=note]{NOTE: #1}}
\newcommand{\todocomment}[1]{\todo[inline,color=comments]{COMMENT: #1}}
\newcommand{\todoaction}[1]{\todo[inline,color=actions]{ACTION: #1}}
\newcommand{\todoalert}[1]{\todo[inline,color=alerts]{ALERT: #1}}


\newcommand{\head}[1]{\chapter{#1}}

\definecolor{principle}{rgb}{0.7,0.9,1.0}
\definecolor{concept}{rgb}{0.8,0.8,0.8}
\definecolor{subconcept}{rgb}{0.9,0.9,0.9}

\newcommand{\principle}[1]{\begin{mdframed}[backgroundcolor=principle!20]\raggedright\textbf{\say{#1}}\end{mdframed}}

\newcommand{\concept}[1]{\begin{mdframed}[backgroundcolor=concept!20]
\raggedright
#1
\end{mdframed}}
\newcommand{\knowledge}[1]{\begin{mdframed}[backgroundcolor=concept!20]\raggedright#1\end{mdframed}}
\newcommand{\subconcept}[1]{\begin{mdframed}[backgroundcolor=subconcept!20]\raggedright{{#1}}\end{mdframed}}

\newcommand{\ruled}{\noindent\rule{\textwidth}{1pt}}
\newcommand{\question}[2]{
	\ifx&#2&%
	\section{#1}
	\else
	\section{#1}{\textit{#2}}
	\fi
}

\newcommand{\unknown}{UNKNOWN}
\newcommand{\staranswer}[5]{
\noindent{\subsection{#1}} 
\noindent\textbf{Situation:} #2 \\
\noindent\textbf{Task: }#3  \\
\noindent\textbf{Action: }#4 \\
\noindent\textbf{Result: }#5 
}

\newcommand{\issue}[4]{
	\noindent{\subsection{#1}} 
	\noindent\textbf{Resolution: }#2  \\
	\noindent\textbf{Detail: }#3
	\ifx&#4&%
	\else
	#4	
	\fi
}

\newcommand{\items}[1]{
	\begin{itemize}[noitemsep]
		#1
	\end{itemize}
}


\newcommand{\imagefig}[2]{
	\centering
	\begin{minipage}[!t]{.49\textwidth}
		\vspace{0pt}
		\centering
		\ifx&#1&%
		\else
		\includegraphics[width=1\linewidth]{images/#1}
		\captionsetup{labelfont=bf,singlelinecheck=on,justification=raggedright}
		\caption{#2}
		\label{fig:#1}
		\fi
	\end{minipage}%
}


\newcommand{\imagefigsingle}[2]{
	\begin{figure}[!h]
		\centering
		\includegraphics[width=0.9\linewidth]{images/#1}
		\captionsetup{labelfont=bf,singlelinecheck=on}
		\caption{#2}
		\label{fig:#1}
	\end{figure}
}

\newcommand{\imagefigdual}[4]{
	\ifx&#3&%
	\begin{figure}[!h]
		\raggedright
		\imagefig{#1}{#2}
		\imagefig{}{}
	\end{figure}
	\else
	\begin{figure}[!h]
		\imagefig{#1}{#2}
		\imagefig{#3}{#4}
	\end{figure}
	\fi
}

\newcommand{\imagefigquad}[8]{
	\begin{figure}[!t]
		\imagefig{#1}{#2}
		\imagefig{#3}{#4}
		\ifx&#7&%
		\raggedright
		\imagefig{#5}{#6}
		\else
		\imagefig{#5}{#6}
		\imagefig{#7}{#8}
		\fi
	\end{figure}
}

\newcommand{\dualtableimage}[6]{
	\begin{figure}[t!]
		\centering
		\begin{minipage}{.48\textwidth}
			\vspace{0pt}
			\small
			\begin{tabular}{#1}
				\hline
				#2
				\hline
			\end{tabular}
			\captionsetup{labelfont=bf,singlelinecheck=on,justification=raggedright}
			\captionof{table}{Table #3}
			\label{table:#4}
		\end{minipage}
		\hfill
		\begin{minipage}{.48\textwidth}
			\vspace{0pt}
			\includegraphics[width=1\linewidth]{images/#5}
			\captionsetup{labelfont=bf,singlelinecheck=on,justification=raggedright}
			\captionof{figure}{#6}
			\label{fig:#5}
		\end{minipage}
	\end{figure}
}
